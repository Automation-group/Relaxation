\part{Система команд БШД}

\begin{tabular}{|c|c|p{10cm}|}
\hline
\multicolumn{3}{|c|}{\textbf{Система команд БШД}} \\
\hline
\bfseries Команда & \bfseries Ответ & \bfseries Назначение\\
\hline
RMx & RMx & Запуск мотора на x-шагов (1 $\leq$ x $\leq$ 4100000000).
 Если параметр x отсутствует, то мотор вращается до получения команды СТОП (SM).\\
\hline
SM & SM & Остановить мотор (СТОП).\\
\hline
SDx & SDx & Задать направление вращения, x = F --- по часовой стрелке; x = B --- против часовой стрелке.\\
\hline
EM & EM & Включить обмотки двигателя.\\
\hline
DM & DM & Выключить обмотки двигателя.\\
\hline
GE & GEx & Получить состояние мотора, где x = D --- выключен;
 x = R --- вращается; x = S --- остановлен.\\
\hline
GD & GDx & Получить направление вращения, где x = F --- по часовой стрелке;
 x = B --- против часовой стрелке.\\
\hline
SCx & SCx & Установить значение счетчика шагов. x --- число шагов (-4 100 000 000 $\leq$ x $\leq$ 4 100 000 000).\\
\hline
GC & GCx & Получить значение счетчика шагов. x --- число шагов.\\
\hline
- & EVDU & Сработал верхний концевик.\\
\hline
- & EVDD & Сработал нижний концевик.\\
\hline
- & EVUU & Отпущен верхний концевик.\\
\hline
- & EVUD & Отпущен нижний концевик.\\
\hline
- & EVUF & Перегрузка.\\
\hline
- & EVUT & Перегрев.\\
\hline
- & EVRD & Команда вызывается всегда, когда двигатель остановился.\\
\hline
SFx & SFx & Установить частоту мотора, x --- частота в тысячных долях герца (1 Гц $\leq$ f $\leq$ 32 000 Гц).
Шаг по частоте равен 0.001 Гц. По умолчанию частота 20 Гц. (1000 $\leq$ x $\leq$ 32000000 )\\
\hline
SUx & SUx & Установить соответствие врещения двигателя движению пуансона.
     1 - Пуансон перемещается вверх при врещении двигателя по часовой стрелке.
     0 - Пуансон перемещается вверх при врещении двигателя против часовой стрелки.\\
\hline
\end{tabular}

\begin{tabular}{|c|c|p{10cm}|}
\hline
\multicolumn{3}{|c|}{\textbf{Система команд БШД}} \\
\hline
\bfseries Команда & \bfseries Ответ & \bfseries Назначение\\
\hline
GU & GUx & Прочитать соответствие врещения двигателя движению пуансона.
     1 - Пуансон перемещается вверх при врещении двигателя по часовой стрелке.
     0 - Пуансон перемещается вверх при врещении двигателя против часовой стрелки.\\
\hline
GMF & GMFx & x - 1 перегрев, 0 перегрева нет.\\
\hline
GMT & GMTx & x - 1 перегрузка, 0 перегрузки нет.\\
\hline
GF & GFx & Вернуть частоту мотора, x --- частота в герцах.\\
\hline
GT & GTxy & Вернуть состояние концевиков. x --- верхний концевик, 
 U --- свободен, D – нажат; y --- нижний концевик, U --- свободен, D --- нажат.\\
\hline
\multicolumn{3}{r}{Все команды и ответы заканчиваются '\textbackslash n' символом.} \\
\end{tabular}
\\
Настройки RS232:
\begin{itemize}
\item скорость прередачи --- 115200 Бит/с;
\item стоповых битов --- 2;
\item проверка четности --- нет;
\item бит данных --- 8.
\end{itemize}
