\begin{tabular}{|c|c|p{10cm}|}
\hline
\multicolumn{3}{|c|}{\textbf{Команды АСС}} \\
\hline
Команда & Ответ & Описание \\
\hline
GM & GMx & Произвести единичное измерение, где x --- код АЦП.\\
\hline
RCx & --- & Запустить измерения в непрерывном режиме, где x --- интервал
измерений в мс. 10 $\leq$ x $\leq$ 18000 мс.\\
\hline
--- & GRx & Результат измеренния в непрерывном режиме, где x --- код АЦП.\\
\hline
SC & --- & Остановить измерения в непрерывном режиме.\\
\hline
GS & GSx & Получить состояние прибора, где x=D --- тензодатчик не подключен, x=C
--- запущен непрерывный режим, x=W --- прибор готов к работе.\\
\hline
SOx,y & SOx,y & Установить допустимые пределы силы.
Если код АЦП выходит из пределов: x $\leq$ код АЦП $\leq$ y --- устанавливается
сигнал перегрузка.\\
\hline
SO & SOx,y & Получить допустимые пределы силы.
Если код АЦП выходит из пределов: x $\leq$ код АЦП $\leq$ y --- устанавливается
сигнал перегрузка.\\
\hline
GO & GOx & Получить состояние перегрузки, где x $\neq$ 0 --- перегрузка.\\
\hline
ID & IDx & Прочитать идентификатор прибора, где x --- идентификатор, Md5 от 'ТУБ
версия 2' в кодировке UTF-8 --- '2f8771d5ebf55e0983210304c6d5197e'.\\ \hline
\multicolumn{3}{r}{Все команды и ответы заканчиваются символом '\textbackslash n'.} \\
\end{tabular}

Настройки RS232:
\begin{itemize}
\item скорость прередачи --- 115200 Бит/с;
\item стоповых битов --- 2;
\item проверка четности --- нет;
\item бит танных --- 8.
\end{itemize}